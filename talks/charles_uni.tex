\documentclass[xcolor=dvipsnames]{beamer}
\usepackage{graphicx}
\usepackage{subcaption}
\usepackage{animate}
\usepackage{amsmath}
\usepackage{hyperref}

\addtobeamertemplate{navigation symbols}{}{%
    \usebeamerfont{footline}%
    \usebeamercolor[fg]{footline}%
    \hspace{1em}%
    \insertframenumber/\inserttotalframenumber
}

\title{Evaluating complexity  in cellular automata}
\author{Hugo Cisneros}
\date{May 16th 2019}

\begin{document}
\maketitle

\begin{frame}{Definition}
  \begin{block}{Cellular automaton (CA)}
    A mathematical model composed of \alert{elementary components} (cells) that
    are updated in discrete time steps according to \alert{local rules}.


    The cells can take $k$ possible values (states).
  \end{block}

  Although very simples, cellular automata exhibit very interesting and unusual
  properties.

  CAs are defined by a local transition table.

  \vfill
  \begin{minipage}[h]{.3\linewidth}
    \textbf{1D Automaton:}
    \vfill

    \includegraphics[width=.8\linewidth]{../figures/1d_neigh}
  \end{minipage}
  \hfill
  \begin{minipage}[h]{.6\linewidth}
    \textbf{2D Automaton:}
    \vfill

    \includegraphics[width=.4\linewidth]{../figures/2d_moore}
    \hfill
    \includegraphics[width=.4\linewidth]{../figures/2d_von_neumann}
  \end{minipage}


\end{frame}

\begin{frame}{Wolfram's classification}
  Wolfram has studied extensively the 1D elementary CA and created a
  classification of the 256 possible rules into 4 classes.
  \vfill

  \begin{description}
  \item[Class 1]<1-| alert@1> Fixed homogeneous state is reached
  \item[Class 2]<1-| alert@2> A pattern of periodic regions is produced
  \item[Class 3]<1-| alert@3> A chaotic aperiodic pattern is produced
  \item[Class 4]<1-| alert@4> Complex localized structures are generated
  \end{description}

  \begin{minipage}[h]{.04\linewidth}
    \includegraphics[width=\linewidth]{../figures/time.pdf}
  \end{minipage}
  \begin{minipage}[h]{.9\linewidth}
    \begin{figure}
      \begin{subfigure}{.49\linewidth}
        \centering \includegraphics<1>[width=.9\linewidth]{../rule_gif/1d_8}
        \includegraphics<2>[width=.9\linewidth]{../rule_gif/1d_4}
        \includegraphics<3>[width=.9\linewidth]{../rule_gif/1d_45}
        \includegraphics<4>[width=.9\linewidth]{../rule_gif/1d_110}
        \caption{Rule \only<1>{8}\only<2>{4}\only<3>{45}\only<4>{110}}
      \end{subfigure}
      \only<1,2,3>{\begin{subfigure}{.49\linewidth} \centering
          \includegraphics<1>[width=.9\linewidth]{../rule_gif/1d_253}
          \includegraphics<2>[width=.9\linewidth]{../rule_gif/1d_37}
          \includegraphics<3>[width=.9\linewidth]{../rule_gif/1d_30}
          \caption{Rule \only<1>{253}\only<2>{37}\only<3>{30}}
        \end{subfigure}}
    \end{figure}
  \end{minipage}
  \only<4>{\textbf{Rule 110 is computationally universal.}}
\end{frame}

\begin{frame}{Sensitivity to initial state}
  \begin{figure}[htbp]
    \centering
    \begin{subfigure}{.3\linewidth}
      \includegraphics[width=\linewidth]{../figures/rule_22_chaos}
      \caption{Chaotic - (Class 3)}
    \end{subfigure}
    \begin{subfigure}{.3\linewidth}
      \includegraphics[width=\linewidth]{../figures/rule_22_order}
      \caption{Ordered - (Class 2)}
    \end{subfigure}
    \begin{subfigure}{.3\linewidth}
      \includegraphics[width=\linewidth]{../figures/rule_22_stop}
      \caption{Homogenous - (Class 1)}
    \end{subfigure}
    \caption{Rule 22 - Random initial state top left 12 cells. Size: 64 cells,
      ran for 128 steps}
  \end{figure}


\end{frame}


\begin{frame}{Compression based study of 1D CAs}
  \begin{block}{Idea:}
    Use the compressed length of a CA state as a proxy for its \emph{complexity}.
  \end{block}
  \vfill

  \only<1>{ Many definitions of complexity: here the Kolmogorov complexity is
    \alert{constant} for a given rule.

    We are looking for a more \emph{qualitative} interpretation of complexity,
    similar to what human beings perceive.
  }


  \only<2>{
  A CA state is represented as a string of 0s and 1s (or something else for more
  states) that can be fed to a compression algorithm (gzip in the rest of the
  presentation).

  \vfill
  \begin{figure}[htbp]
    \centering
    \includegraphics[width=0.8\linewidth]{../figures/compress}
    \label{fig:string_conv}
  \end{figure}
  \vfill

  Examples:
  \begin{figure}[htbp]
    \begin{subfigure}{.7\linewidth}
      \centering
      \includegraphics[width=\linewidth]{../figures/ordered_state_1d}
    \end{subfigure} \quad $\rightarrow$ \quad $\ell = 13$


    \begin{subfigure}{.7\linewidth}
      \centering
      \includegraphics[width=\linewidth]{../figures/random_state_1d}
    \end{subfigure} \quad $\rightarrow$ \quad $\ell = 34$
  \end{figure}
  }
\end{frame}


\begin{frame}{Compressed length for single-cell initialization}
  \begin{figure}[h]
    \centering
    \includegraphics[width=.5\linewidth]{../figures/one.png}
    \caption{Single cell activated in initial configuration, evolve an automaton
    of size 1024 for 512 timesteps}
    \label{fig:1_cell}
  \end{figure}

  $\rightarrow$ \textbf{Obtained classification matches exactly the Wolfram
    ``manual'' classification (originally observed by Zenil, 2010)}
\end{frame}


\begin{frame}{Influence of the compression algorithm}
  \begin{figure}[htbp]
    \centering
    \includegraphics[width=.6\linewidth]
    {../figures/rules_lowcpx_800bits800ts_rand.pdf}

    \caption{Temporal evolution of compressed length for 3 rules. Comparison of
      gzip with PAQ}
    \label{fig:temp_evol}
  \end{figure}

\end{frame}


\begin{frame}{2D CAs}

  With 2D CAs, the number of rules is much higher: $2^{512}$ rules total,
  $2^{102}$ if we require the rules to have all symmetries.
  \vfill

  Gets even bigger if we add more states and/or larger neighborhoods.
  \vfill

  \begin{itemize}
    \item There is no chance of sampling all the rules (not even a
      \emph{significant} portion of it)
    \item Some parts of this space might be more interesting than others
  \end{itemize}
  \vfill

  $\rightarrow$ \textbf{We need a way of guiding this search towards
    \alert{interesting} rules}

\end{frame}


\begin{frame}{Compressed length repartition}

  Sample rules at random and compress the state as an ``unrolled string''.

  \begin{figure}[htbp]
    \centering
    \includegraphics[width=.5\linewidth]{../figures/2d_compressed_distrib.pdf}
    \caption{Compressed length distribution for $k=2$, 2D CAs. Grid size is $256
      \times 256$, automata are ran for 1000 time steps.}
    \label{fig:comp_length_2d}
  \end{figure}

\end{frame}


\begin{frame}{Extreme cases --- Illustration}

  \begin{figure}[htbp]
    \centering
    \begin{subfigure}{.49\linewidth}
      \centering
      \animategraphics[loop,controls,width=.9\linewidth]{5}
      {../rule_gif/pngs/13463141642615599940-}{0}{16}
      \caption{High compressed length}
    \end{subfigure}
    \begin{subfigure}{.49\linewidth}
      \centering
      \animategraphics[loop,controls,width=.9\linewidth]{5}
      {../rule_gif/pngs/3891041186864560738-}{0}{18}
      \caption{Low compressed length}
    \end{subfigure}
  \end{figure}

\end{frame}


\begin{frame}{Intermediate cases --- Illustration}

  \begin{figure}[htbp]
    \centering
    \begin{subfigure}{.49\linewidth}
      \centering
      \animategraphics[loop,controls,width=.9\linewidth]{5}
      {../rule_gif/pngs/1461082546904894069-}{0}{19}
      \caption{Compressed length = 2914}
    \end{subfigure}
    \begin{subfigure}{.49\linewidth}
      \centering
      \animategraphics[loop,controls,width=.9\linewidth]{5}
      {../rule_gif/pngs/16172727384030072437-}{0}{16}
      \caption{Compressed length = 6753}
    \end{subfigure}
  \end{figure}

\end{frame}


\begin{frame}{Beyond compressed length}
  Compressed length is not the right metric for 2D CAs
  \begin{itemize}
    \item Most rules are at the extremes of the graph.
    \item \emph{Interesting} rules might have very different compressed lengths,
      what matters is the \alert{dynamic} of this complexity.
  \end{itemize}

\end{frame}


\begin{frame}{Joint compression score}

  To measure the \alert{stability} of patterns for an evolving 2D CA, we use the
  joint compression, i.e. the compressed length of the concatenation of two
  steps relatively far apart in time.

  \begin{figure}[htbp]
    \centering
    \includegraphics[width=.6\linewidth]{../figures/joint_compression}
    \label{fig:joint_comp}
  \end{figure}

\end{frame}

\begin{frame}{Results on 2D CA rules}
  The score we compute is $\dfrac{C_1 + C_2}{C_{12}}$. Distribution on the
  histogram below.

  \begin{minipage}[h]{.49\linewidth}
    \begin{figure}[htbp]
      \centering
      \includegraphics[width=\linewidth]{../figures/hist_score2.pdf}
      \caption{Joint compression score for 13000 random rules}
      \label{fig:joint_hist}
    \end{figure}
  \end{minipage}
  \begin{minipage}[h]{.49\linewidth}
    Two parts:
    \begin{itemize}
    \item A large portion of rules that have \textbf{very high compressed length
        and no structure} (low joint compression score).
    \item Other group ($\sim$1\%)that seems to have \textbf{much more structure}
      (although not all rules do and not all ``structured rules'' exhibit
      interesting behavior).
    \end{itemize}
  \end{minipage}

\end{frame}

\begin{frame}{Example rules --- 2 states}
  \begin{figure}[htbp]
    \centering
    \begin{subfigure}{.32\linewidth}
      \centering
      \animategraphics[loop,controls,width=\linewidth]{5}
      {../rule_gif/pngs/5526130398097930830-}{0}{19}
    \end{subfigure}
    \begin{subfigure}{.32\linewidth}
      \centering
      \animategraphics[loop,controls,width=\linewidth]{5}
      {../rule_gif/pngs/6349956867804255879-}{0}{18}
    \end{subfigure}
    \begin{subfigure}{.32\linewidth}
      \centering
      \animategraphics[loop,controls,width=\linewidth]{5}
      {../rule_gif/pngs/6116844074278280215-}{0}{16}
    \end{subfigure}
  \end{figure}
\end{frame}

\begin{frame}{3 States --- 2D CAs}
  \begin{figure}[htbp]
    \centering
    \begin{subfigure}{.49\linewidth}
      \centering
      \includegraphics[width=.9\linewidth]{../figures/hist_score3.pdf}
      \caption{Score histogram for 3-states automata}
    \end{subfigure}
    \begin{subfigure}{.49\linewidth}
      \centering
      \includegraphics[width=.9\linewidth]{../figures/2d_compressed_distrib3}
      \caption{Compressed length histogram for 3-states automata}
    \end{subfigure}
  \end{figure}

  Similar distributions $\rightarrow$ if this can be generalized, this would be
  a first toward creating a systematic approach for finding interesting rules.
\end{frame}

\begin{frame}{Example rules --- 3 states}
  \begin{figure}[htbp]
    \centering
    \begin{subfigure}{.49\linewidth}
      \centering
      \animategraphics[loop,controls,width=\linewidth]{5}
      {../rule_gif/pngs/2656185470595024321_3-}{0}{19}
    \end{subfigure}
    \begin{subfigure}{.49\linewidth}
      \centering
      \animategraphics[loop,controls,width=\linewidth]{5}
      {../rule_gif/pngs/6395857399678607410_3-}{0}{18}
    \end{subfigure}
  \end{figure}
\end{frame}

\begin{frame}{Further work}
  \begin{itemize}
    \item Study the influence of the initial state.
    \item Add some input/output capabilities ?
    \item Refine the metric or find some other ?
  \end{itemize}

\end{frame}


\end{document}